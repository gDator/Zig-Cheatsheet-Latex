\documentclass[10pt,a4paper]{article}
\usepackage[utf8]{inputenc}
\usepackage[landscape,margin=1cm]{geometry}
\usepackage[english]{babel}
\usepackage[latte]{catppuccinpalette}
\usepackage{amsmath}
\usepackage{amssymb}
\usepackage[svgnames]{xcolor}
\usepackage{graphicx}
\usepackage{fullpage}
\usepackage{minted}
\usepackage{xparse}
\usemintedstyle[zig]{paraiso-light}
\usepackage{hyperref}
\usepackage{fancyhdr}% http://ctan.org/pkg/fancyhdr
% colour themes to come. KnitR?

%-------------------------

\usepackage[default]{raleway}
\usepackage[T1]{fontenc}
\usepackage{fontawesome5}
\usepackage{pdfx}
\usepackage{enumitem}
\usepackage{lipsum}
\usepackage{xcolor}
\definecolor{customcolor}{HTML}{616AC5}
\definecolor{alert}{HTML}{CD5C5C}
\definecolor{w3schools}{HTML}{4CAF50}
\definecolor{subbox}{gray}{0.60}
\definecolor{codecolor}{HTML}{FFC300}
\colorlet{xx}{customcolor}

%--------------------------Editor mode.

\usepackage
[citestyle=authoryear,
sorting=nty,	  		%Sorts bibliography by year, name, title
autocite=footnote, 		%Autocite command generates footnotes
autolang=hyphen, 		
mincrossrefs=1, 	
backend=biber]
{biblatex}

\DeclareFieldFormat{postnote}{#1}
\DeclareFieldFormat{multipostnote}{#1}
\DeclareAutoCiteCommand{footnote}[f]{\footcite}{\footcites}
\bibliography{literature}
%----------------------------------------
%--------------------------------------------------------------------------------
\usepackage[many]{tcolorbox}

\tcbuselibrary{most,listingsutf8,minted, skins, breakable}

\tcbset{tcbox width=auto,left=1mm,top=1mm,bottom=1mm,
	right=1mm,boxsep=1mm,middle=1pt}

\newenvironment{mycolorbox}[2]{%
	\begin{tcolorbox}[grow to left by=-1em,grow to right by=-1em,capture=minipage,fonttitle=\large\bfseries, enhanced jigsaw,boxsep=1mm,colback=#1!30!white,on line,tcbox width=auto, toptitle=0mm,colframe=#1,opacityback=0.7,nobeforeafter,title=#2, breakable]%
	}{\end{tcolorbox}}

\newenvironment{subbox}[2]{%
	\begin{tcolorbox}[capture=minipage,fonttitle=\normalsize\bfseries, enhanced jigsaw,boxsep=1mm,colback=#1!30!white,on line,tcbox width=auto,left=0.3em,top=1mm, toptitle=0mm,colframe=#1,opacityback=0.7,nobeforeafter,title=#2]\footnotesize %
	}{\normalsize\end{tcolorbox}\vspace{0.1em}}

\newenvironment{multibox}[1]{%
	\begin{tcbraster}[raster columns=#1,raster equal height,nobeforeafter,raster column skip=1em,raster left skip=1em,raster right skip=1em]}{\end{tcbraster}}

\newenvironment{textbox}[1]{\begin{mycolorbox}{customcolor}{#1}}{\end{mycolorbox}}

%-------------------------------
\newtcblisting{codebox}[3]{colback=#3!5,colframe=#3!80!black,listing only, 
	minted options={numbers=left,fontsize=\tiny,breaklines,autogobble,numbersep=3mm},
	left=5mm,enhanced,
	title=#2, fonttitle=\bfseries,
	listing engine=minted,minted language=#1}
%--------------------------------------------------------------------------------
\newcommand{\punkti}{~\lbrack\dots\rbrack~}

\renewenvironment{quote}
{\list{\faQuoteLeft\phantom{ }}{\rightmargin\leftmargin}%
	\item\relax\scriptsize\ignorespaces}
{\unskip\unskip\phantom{xx}\faQuoteRight\endlist}


%--------------------------------------------------------------------------------
\newcommand{\bgupper}[3]{\colorbox{#1}{\color{#2}\huge\bfseries\MakeUppercase{#3}}}
\newcommand{\bg}[3]{\colorbox{#1}{\bfseries\color{#2}#3}}

\newcommand{\mycommand}[2]{{\ttfamily\detokenize{#1}}~\dotfill{}~{\footnotesize #2}\\}
\newcommand{\sep}{{\scriptsize~\faCircle{ }~}}


\newcommand{\bggreen}[1]{\medskip\bgupper{w3schools}{black}{#1}\\[0.5em]}
\newcommand{\green}[1]{\smallskip\bg{w3schools}{white}{#1}\\}
\newcommand{\red}[1]{\smallskip\bg{alert}{white}{#1}\\}

\usepackage{multicol}
\setlength{\columnsep}{30pt}

\setlength{\parindent}{0pt}
\pagestyle{empty}

\usepackage{csquotes}

\newcommand{\loremipsum}{Lorem ipsum dolor sit amet.}
\BeforeBeginEnvironment{minted}{\vspace{-2mm}}
\pagenumbering{arabic}


\fancyhf{}% Clear header/footer
\fancyfoot[R]{\thepage}% \fancyfoot[R]{\thepage}
\renewcommand{\headrulewidth}{0pt}%
\renewcommand{\footrulewidth}{0.4pt}% Default \footrulewidth is 0pt
\pagestyle{fancy}

%--------------------------------------------------------------------------------
\begin{document}
	\small
		\begin{multicols*}{3}

		\LARGE\centering\textcolor{CtpPeach}{\textbf{Zig 0.13}}
		\vspace{2cm}
		\normalsize
		\begin{mycolorbox}{CtpPeach}{Entry Point}
			\begin{minted}[autogobble, breaklines]{zig}
			pub fn main() void {
				std.debug.print("Hello world!\n", .{});
			}
			\end{minted}
		\end{mycolorbox}
\begin{mycolorbox}{CtpSky}{Basic}
	\textbf{Variable}
	\begin{minted}[autogobble, breaklines, xleftmargin=1cm]{zig}
		var n: u8 = 50;
	\end{minted}

	\textbf{Constant}
	\begin{minted}[autogobble, breaklines, xleftmargin=1cm]{zig}
		const pi: u32 = 314159;
	\end{minted}

	\textbf{Array}
	\begin{minted}[autogobble, breaklines, xleftmargin=1cm]{zig}
	var array = [_]u8{ 1, 0b0000010, 0x03, 0o04 };
	\end{minted}

	\textbf{Array addition}
	\begin{minted}[autogobble, breaklines, xleftmargin=1cm]{zig}
		var array_result = array_a ++ array_b;
	\end{minted}

	\textbf{Array repetition}
	\begin{minted}[autogobble, breaklines, xleftmargin=1cm]{zig}
	var array_result = array_a ** 3;
	\end{minted}

	\textbf{Pointer}
	\begin{minted}[autogobble, breaklines, xleftmargin=1cm]{zig}
		const pointer: *u8 = &n;
	\end{minted}

	\textbf{	Pointer dereferencing}
	\begin{minted}[autogobble, breaklines, xleftmargin=1cm]{zig}
	var n: u8 = pointer.*;
	\end{minted}

	\textbf{Pointer access}
	\begin{minted}[autogobble, breaklines, xleftmargin=1cm]{zig}
		var n: u8 = struct_pointer.a;
%	\end{minted}
	\textbf{Slice}
	\begin{minted}[autogobble, breaklines, xleftmargin=1cm]{zig}
		var slice = array[0..3]; // 1, 2, 3
	\end{minted}
	\textbf{Sentinel} 
	\begin{minted}[autogobble, breaklines, xleftmargin=1cm]{zig}
		const ptr: [*:0]u32 = &nums;
	\end{minted}
	\textbf{Tuple (anonymous struct)}
	\begin{minted}[autogobble, breaklines, xleftmargin=1cm]{zig}
		const tuple = .{true, false, @as(i32, 42), @as(f32, 3.141592), };
	\end{minted}
	\textbf{Anonymous list} 
	\begin{minted}[autogobble, breaklines, xleftmargin=1cm]{zig}
	const hello: [5]u8 = .{ 'h', 'e', 'l', 'l', 'o' };
%	\end{minted}
	\textbf{Bit manipulation}
	\begin{minted}[autogobble, breaklines, xleftmargin=1cm]{zig}
		const res1 = numOne >> 4;
		const res2 = numOne << 4;
		const res3 = numOne & numTwo;
		const res4 = numOne | numTwo;
		const res5 = numOne ^ numTwo;
	\end{minted}
\end{mycolorbox}


\begin{mycolorbox}{CtpCrust}{Datatypes}
			\textbf{Unsigned Integer} \mintinline[autogobble, breaklines, xleftmargin=1cm,breakafter={/}]{zig}{u8, u16, u32, u64}\\
			\textbf{Integer} \mintinline[autogobble, breaklines, xleftmargin=1cm,breakafter={/}]{zig}{ i8, i16, i32, i64}\\
			\textbf{Float} \mintinline[autogobble, breaklines, xleftmargin=1cm,breakafter={/}]{zig}{f16, f32, f64, f80, f128}\\
			\textbf{String} \mintinline[autogobble, breaklines, xleftmargin=1cm,breakafter={/}]{zig}{[_]u8}\\
			\textbf{Bool} \mintinline[autogobble, breaklines, xleftmargin=1cm,breakafter={/}]{zig}{bool}\\
			\textbf{Pointer} \mintinline[autogobble, breaklines, xleftmargin=1cm,breakafter={/}]{zig}{*u8}\\
			\textbf{Pointer to Constant} \mintinline[autogobble, breaklines, xleftmargin=1cm,breakafter={/}]{zig}{*const u8} \\
			\textbf{Slice} \mintinline[autogobble, breaklines, xleftmargin=1cm,breakafter={/}]{zig}{[]const u8, []u8}\\
			\textbf{Many-Pointer}(length is lost) \mintinline[autogobble, breaklines, xleftmargin=1cm,breakafter={/}]{zig}{[*]const u8, [*]u8}\\
\end{mycolorbox}

\begin{mycolorbox}{CtpBase}{Strings}
	\textbf{String} \mintinline[autogobble, breaklines, xleftmargin=1cm,breakafter={/}]{zig}{[_]u8}\\
	\textbf{Multiline String} \begin{minted}[autogobble, breaklines, xleftmargin=1cm,breakafter={/}]{zig}
		var string = 
		  \\Line 1
		  \\Line 2
	\end{minted}
\end{mycolorbox}

\begin{mycolorbox}{CtpTeal}{Unions}
	\textbf{Definitions}
	\begin{minted}[autogobble, breaklines, xleftmargin=1cm,breakafter={/}]{zig}
		const Data = union {
		  index: u16,
		  link: bool,
		};
	\end{minted}

	\textbf{Access/Reassignment}
	\begin{minted}[autogobble, breaklines, xleftmargin=1cm,breakafter={/}]{zig}
		var node = Data{ .link = true };     //OK
		const node_n = node.index;           //Crash
		node = Data{ .index = 1};            //OK
	\end{minted}

		\textbf{Tagged Union}
	\begin{minted}[autogobble, breaklines, xleftmargin=1cm,breakafter={/}]{zig}
	const Data = union(DataType){
	  index: u16,
	  link: bool,
	};
	const Data = union(enum){
	  index: u16,
	  link: bool,
	};
	\end{minted}

		\textbf{Unpack Tagged Union} (Values in switch statement are enum values)
	\begin{minted}[autogobble, breaklines, xleftmargin=1cm,breakafter={/}]{zig}
		switch (node) {
		  .link => |l| ...,
		  .index => |i|,
		  inline else => |x| ...,
		}
	\end{minted}
\end{mycolorbox}


\begin{mycolorbox}{CtpGreen}{Optionals}
	Optional can be value or null
	\textbf{Defintion}
	\begin{minted}[autogobble, breaklines, xleftmargin=1cm,breakafter={/}]{zig}
		const value: ?u8 = null;
	\end{minted}

	\textbf{Assignment}(0 if value is null)
	\begin{minted}[autogobble, breaklines, xleftmargin=1cm,breakafter={/}]{zig}
		const value_b: u8 = value orelse 0;
	\end{minted}
	
	\textbf{Forcing value to be not null}
	\begin{minted}[autogobble, breaklines, xleftmargin=1cm,breakafter={/}]{zig}
		const value_b: u8 = value orelse unreachable;
	\end{minted}

	\textbf{Extraction}
	\begin{minted}[autogobble, breaklines, xleftmargin=1cm,breakafter={/}]{zig}
		const value_b: u8 = value.?;
	\end{minted}
\end{mycolorbox}

\begin{mycolorbox}{CtpRosewater}{Error}
	\textbf{Error Definition}
	\begin{minted}[autogobble, breaklines, xleftmargin=1cm]{zig}
		const SpecialError = error{
			NoNumber,
			DivisionByZero,
			InfError,
		};
	\end{minted}

	\textbf{Error Union} Variable can be either error or datatype
	\begin{minted}[autogobble, breaklines, xleftmargin=1cm]{zig}
		var number_or_error: SpecialError!u8 = 5;
	\end{minted}

	\textbf{Error Catching}
	\begin{minted}[autogobble, breaklines, xleftmargin=1cm, breakafter={.}]{zig}
		return funcWithError(n) catch |err| {
			if (err == SpecialError.DivisonByZero) {
				return 0;
			}
			return err;
		};
	\end{minted}

	\textbf{Standard Error Catching}
	\begin{minted}[autogobble, breaklines, xleftmargin=1cm]{zig}
		funcWithError(n) catch |err| return err;
		try funcWithError(n);
	\end{minted}

	\textbf{Error Extraction}
	\begin{minted}[autogobble, breaklines, xleftmargin=1cm]{zig}
		const n = funcWithError();
		if (n) |value| {
		} else |err| switch (err) {}
	\end{minted}

	\textbf{Error Packaging}
	\begin{minted}[autogobble, breaklines, xleftmargin=1cm]{zig}
		const SpecialError = SpecialErrorA || SpecialErrorB;
	\end{minted}
\end{mycolorbox}


\begin{mycolorbox}{CtpLavender}{Enums}
	\textbf{Definition}
	\begin{minted}[autogobble, breaklines, xleftmargin=1cm,breakafter={/}]{zig}
		const Fruit = enum { APPLE, BANANA, STRAWBERRY,  };
		const Fruit = enum(u8) { APPLE = 1, BANANA = 2,  };
	\end{minted}
\end{mycolorbox}


\begin{mycolorbox}{CtpFlamingo}{Structs}
	\textbf{Defintion}
	\begin{minted}[autogobble, breaklines, xleftmargin=1cm,breakafter={/}]{zig}
		const Picture = struct {
		  width: u32,
		  height: u32,
		  data: [_]u32,
		};
	\end{minted}

		\textbf{Declaration}
	\begin{minted}[autogobble, breaklines, xleftmargin=1cm,breakafter={/}]{zig}
		var pic = Picture {
		  .width = 10,
		  .height = 10,
		  .data = {...},
		};
	\end{minted}

		\textbf{Access} \mintinline[autogobble, breaklines, xleftmargin=1cm,breakafter={/}]{zig} {pic.data = {...};} \\
		\textbf{Method}
	\begin{minted}[autogobble, breaklines, xleftmargin=1cm,breakafter={/}]{zig}
		const Picture = struct {
		  width: u32,
		  height: u32,
		  pub fn empty() Picture {
		    ...
		  }
		  pub fn mirrorX(self: *Picture ) void {
		    ...
		  }
		};
		Picture.empty();
		pic.mirrorX();
	\end{minted}

		\textbf{Anonymous struct}
	\begin{minted}[autogobble, breaklines, xleftmargin=1cm,breakafter={/}]{zig}
		fn Circle(comptime T: type) type {
		  return struct {
		    center_x: T,
		    center_y: T,
		    radius: T,
		  };
		}
	\end{minted}
\end{mycolorbox}



\begin{mycolorbox}{CtpSapphire}{Flow Control}
		\textbf{If Statement}
	\begin{minted}[autogobble, breaklines, xleftmargin=1cm,breakafter={/}]{zig}
		if (foo) {
		  std.debug.print("True!\n", .{});
		} else {
		std.debug.print("False!\n", .{});
		}
	\end{minted}

		\textbf{If Assignment}
	\begin{minted}[autogobble, breaklines, xleftmargin=1cm,breakafter={/}]{zig}
		const value: u8 = if (correct) 1 else 2;
	\end{minted}

		\textbf{While loop}
	\begin{minted}[autogobble, breaklines, xleftmargin=1cm,breakafter={/}]{zig}
		while (condition) {}
	\end{minted}

		\textbf{While-Loop with continue expression}
	\begin{minted}[autogobble, breaklines, xleftmargin=1cm,breakafter={/}]{zig}
		while (condition) : (n*=2) {}
	\end{minted}

		\textbf{Continue loop}
	\begin{minted}[autogobble, breaklines, xleftmargin=1cm,breakafter={/}]{zig}
		while (condition) : (n*=2) {
		  if (n % 2 == 0) continue;
		}
	\end{minted}
		\textbf{Break loop }
	\begin{minted}[autogobble, breaklines, xleftmargin=1cm,breakafter={/}]{zig}
		while (true) : (n*=2) {
		  if (n % 2 == 0) break;
		}
	\end{minted}

		\textbf{For-Loop}
	\begin{minted}[autogobble, breaklines, xleftmargin=1cm,breakafter={/}]{zig}
		for (array) |a| {
		  std.debug.print("{}", .{a});
		}
		for (array, 0..) |a, i| {
		  std.debug.print("{} at index {}", .{a, i});
		}
		for (1..20) |n| {...}
		for (hex_nums, dec_nums) |hn, dn| {...}
	\end{minted}

		\textbf{Switch-Statement}
	\begin{minted}[autogobble, breaklines, xleftmargin=1cm,breakafter={/}]{zig}
		switch (c) {
		  1 => std.debug.print("A", .{}),
		  2 => std.debug.print("B", .{}),
		  else => std.debug.print("?", .{})
		}
	\end{minted}

		\textbf{Switch-Assignment}
	\begin{minted}[autogobble, breaklines, xleftmargin=1cm,breakafter={/}]{zig}
		const character: u8 = switch (c) {
		  1 => 'A',
		  2 => 'B',
		  else => '!'
		};
		foo: switch (@as(u8, 1)) {
		  1 => continue :foo 2,
		  2 => continue :foo 3,
		  3 => return,
		  4 => {},
		}
	\end{minted}

		\textbf{Loop-Assignment}
	\begin{minted}[autogobble, breaklines, xleftmargin=1cm,breakafter={/}]{zig}
		const index: ?u8 = for (langs, 0..) |lang, i| {
		  if (lang.len == 2) break i;
		} else null;
	\end{minted}

		\textbf{Lables}
	\begin{minted}[autogobble, breaklines, xleftmargin=1cm,breakafter={/}]{zig}
		const value = outer_loop: for (wave) |v| {
		  for (v.frequency, 0..) |f, i| {
		    if (f.frequency == 0) continue :food_loop;
		  }
		} else wave[0];
	\end{minted}
\end{mycolorbox}

\begin{mycolorbox}{CtpOverlay0}{Functions}
		\textbf{Function}
	\begin{minted}[autogobble, breaklines, xleftmargin=1cm,breakafter={/}]{zig}
		fn func(argument: u32) u32 {
		  return argument;
		}
	\end{minted}

		\textbf{Pass By Reference}
	\begin{minted}[autogobble, breaklines, xleftmargin=1cm,breakafter={/}]{zig}
		fn func(argument: *u32) void{
		  argument = 0;
		}
	\end{minted}

		\textbf{Function with possible Error}
	\begin{minted}[autogobble, breaklines, xleftmargin=1cm,breakafter={/}]{zig}
		fn func(argument: u32) SpecialError!u32 {
		  ... 
		  return u32SpecialError.InfError;
		  ...	
		  return argument;
		}
	\end{minted}

		\textbf{Generic function:}
	\begin{minted}[autogobble, breaklines, xleftmargin=1cm,breakafter={/}]{zig}
		fn makeSequence(comptime T: anytype) void {}
	\end{minted}
\end{mycolorbox}


\begin{mycolorbox}{CtpYellow}{?}
			\textbf{Defer} (Put an statement to end of block)
		\begin{minted}[autogobble, breaklines, xleftmargin=1cm,breakafter={/}]{zig}
			std.debug.print("(", .{});
			defer std.debug.print(") ", .{}); 
			printVector(vec);
		\end{minted}
	
				\textbf{Error defer}
		\begin{minted}[autogobble, breaklines, xleftmargin=1cm,breakafter={/}]{zig}
			fn funcWithError() SpecialError!u32 {
			  // print if function exits with an error:
			  errdefer std.debug.print("failed!\n", .{});
			}
		\end{minted}
	
			\textbf{Unreachable} (Make specific blocks unreachable -> defined program crash)
		\begin{minted}[autogobble, breaklines, xleftmargin=1cm,breakafter={/}]{zig}
			switch (op) {
			  else => unreachable
			}
		\end{minted}
	
			\textbf{Undefined} (Access of undefinied variables is not allowed)
		\begin{minted}[autogobble, breaklines, xleftmargin=1cm,breakafter={/}]{zig}
			var n: u8 = undefined;
		\end{minted}
	
			\textbf{Quoted Identifier} (Put an statement to end of block)
		\mintinline[autogobble, breaklines, xleftmargin=1cm,breakafter={/}]{zig}{@"123_nums"}\\
			\textbf{Tests}
		\begin{minted}[autogobble, breaklines, xleftmargin=1cm,breakafter={/.}]{zig}
			test "add" {
			  try testing.expect(add(41, 1) == 42);
			  try testing.expectError(error.DivisionByZero,                  divide(15, 0));
			}
		\end{minted}
\end{mycolorbox}


\begin{mycolorbox}{CtpRed}{Async}
\end{mycolorbox}

\begin{mycolorbox}{CtpMaroon}{BuiltIn}
	\textbf{Get \lq filename.struct\_name\rq}
	\begin{minted}[autogobble, breaklines, xleftmargin=1cm,breakafter={/}]{zig}
		@TypeOf()
	\end{minted}
	
	\textbf{Typeinfo:}
	\begin{minted}[autogobble, breaklines, xleftmargin=1cm,breakafter={/.}]{zig}
		@typeInfo(Narcissus).@"struct".fields;
		pub const StructField = struct {
			name: []const u8,
			type: type,
			default_value: anytype,
			is_comptime: bool,
			alignment: comptime_int,
		};
	\end{minted}
	
	\textbf{Compile Time logging}
	\begin{minted}[autogobble, breaklines, xleftmargin=1cm,breakafter={/}]{zig}
		@compileLog("Count at compile time: ");
	\end{minted}
	
	\textbf{Compile Time Inheritance(?)} (Returns true if type has a method with given name)
	\begin{minted}[autogobble, breaklines, xleftmargin=1cm,breakafter={/}]{zig}
		@hasDecl(Type, "function");
	\end{minted}
	
	\textbf{Import c header file}
	\begin{minted}[autogobble, breaklines, xleftmargin=1cm,breakafter={/}]{zig}
		const c = @cImport({
			@cInclude("unistd.h");
		});
	\end{minted}
	
	\textbf{Vector}
	\begin{minted}[autogobble, breaklines, xleftmargin=1cm,breakafter={/}]{zig}
		@Vector(len: comptime_int, Element: type)
	\end{minted}
	
	\textbf{Absoulte Value}
	\begin{minted}[autogobble, breaklines, xleftmargin=1cm,breakafter={/}]{zig}
		@abs(value: anytype)
	\end{minted}
	
	\textbf{Transform vector to scalar}
	\begin{minted}[autogobble, breaklines, xleftmargin=1cm,breakafter={/}]{zig}
		@reduce(comptime op: std.builtin.ReduceOp, value: anytype)
	\end{minted}
	
\end{mycolorbox}


\begin{mycolorbox}{CtpSapphire}{Comptime}
		\textbf{Compile time loop}
	\begin{minted}[autogobble, breaklines, xleftmargin=1cm,breakafter={/}]{zig}
		inline for (fields) |field|{
		  if (field.type != void) {
		    print(" {s}", .{field.name});
		  }
		}
	\end{minted}

		\textbf{Compile time variable}
	\begin{minted}[autogobble, breaklines, xleftmargin=1cm,breakafter={/}]{zig}
		comptime var scale: u32 = undefined;
	\end{minted}

		\textbf{Compile time function}
	\begin{minted}[autogobble, breaklines, xleftmargin=1cm,breakafter={/}]{zig}
		fn makeSequence(comptime T: type, comptime size: usize) [size]T {}
	\end{minted}

		\textbf{Compile time block}	\mintinline[autogobble, breaklines, xleftmargin=1cm,breakafter={/}]{zig}{comptime {...}}\\

\end{mycolorbox}

\begin{mycolorbox}{CtpRed}{C Interaction}
\end{mycolorbox}

\begin{mycolorbox}{CtpMauve}{Standard Library}
		\textbf{Import Std}
		\begin{minted}[autogobble, breaklines, xleftmargin=1cm,breakafter={/}]{zig}
		\end{minted}
	
		\textbf{Index of}
		\begin{minted}[autogobble, breaklines, xleftmargin=1cm,breakafter={/}]{zig}
			@import("std").mem.indexOf;
		\end{minted}
	
		\textbf{Std out}
		\begin{minted}[autogobble, breaklines, xleftmargin=1cm,breakafter={/}]{zig}
			const stdout = std.io.getStdOut().writer();
			stdout.print("Hello world!\n", .{});
		\end{minted}
	
			\textbf{Fmt} (Variabletype:filler(Alignment: <>\^)Space)
		\begin{minted}[autogobble, breaklines, xleftmargin=1cm,breakafter={/}]{zig}
			print("{s:*^20}\n", .{"Hello!"});
		\end{minted}
		\textbf{Tokenizer}
		\begin{minted}[autogobble, breaklines, xleftmargin=1cm,breakafter={/}]{zig}
			var it = std.mem.tokenizeAny(u8, poem, " ,;!\n");
		\end{minted}
	
		\textbf{Threads}
		\begin{minted}[autogobble, breaklines, xleftmargin=1cm,breakafter={/}]{zig}
			const handle = try std.Thread.spawn(.{}, thread_function, .{1})
			defer handle.join();
		\end{minted}
	
		\textbf{Filesystem}
		\begin{minted}[autogobble, breaklines, xleftmargin=1cm,breakafter={/}]{zig}
			const cwd: std.fs.Dir = std.fs.cwd()
			cwd.makeDir("dir") catch |e| switch (e) {...}
			var output_dir: std.fs.Dir = cwd.openDir("dir", .{});
			defer output_dir.close();
			const file: std.fs.File = try output_dir.createFile("file.txt", .{});
			defer file.close();
			const byte_written = try file.write("File Opened");
		\end{minted}
\end{mycolorbox}


\begin{mycolorbox}{CtpText}{Allocation}
		\textbf{Arena Allocator}
	\begin{minted}[autogobble, breaklines, xleftmargin=1cm,breakafter={/.}]{zig}
		var arena = std.heap.ArenaAllocator.init(std.heap.page_allocator);
		defer arena.deinit();
		const allocator = arena.allocator();
		const avg: []f64 = try allocator.alloc(f64, 5);
	\end{minted}

		\textbf{General purpose allocater}
	\begin{minted}[autogobble, breaklines, xleftmargin=1cm,breakafter={/.}]{zig}
var gpa = heap.GeneralPurposeAllocator(.{}){};
defer if (gpa.detectLeaks()) log.err("Memory leak detected!", .{});
const alloc = gpa.allocator();
	\end{minted}
\end{mycolorbox}


\begin{mycolorbox}{CtpSapphire}{Build System}
	\textbf{Fetch Dependy}
	\begin{minted}[autogobble, breaklines, xleftmargin=1cm,breakafter={/}]{sh}
		zig fetch --save=vaxis https://github.com/rockorager/libvaxis/archive/refs/tags/v0.5.1.tar.gz
	\end{minted}
\end{mycolorbox}

\begin{mycolorbox}{CtpRed}{Commands}
		\textbf{New Project}
		\begin{minted}[autogobble, breaklines, xleftmargin=1cm,breakafter={/}]{sh}
			zig init
		\end{minted}
\end{mycolorbox}

\end{multicols*}	
\newpage
\begin{mycolorbox}{CtpGreen}{Examples}
	\textbf{Create map with names of enum}
	\begin{minted}[autogobble, breaklines, xleftmargin=1cm,breakafter={/}]{zig}
		pub const entity_names: std.StaticStringMap(EntityType) = EnumStrMap(EntityType);
		pub fn EnumStrMap(V: type) std.StaticStringMap(V) {
			comptime {
				const field = @typeInfo(V).Enum.fields;
				const array_type = struct { [:0]const u8, V };
				var array: [field.len]array_type = undefined;
				for (field, 0..) |f, i| {
					array[i] = array_type{ f.name, @as(V, @enumFromInt(f.value)) };
				}
				return std.StaticStringMap(V).initComptime(array);
			}
		}
	\end{minted}
\end{mycolorbox}

\begin{mycolorbox}{CtpSapphire}{Links/Documentation}
	- \href{https://ziglang.org/documentation/0.14.0/}{Zig Documentation 0.14}\\
	- \href{https://ziglang.org/documentation/0.14.0/std/}{Zig Standard Library Documentation 0.14}\\
	- \href{https://zig.guide/}{Zig Guide}\\
	- \href{https://codeberg.org/ziglings/exercises/#ziglings}{Ziglings examples}\\
	- \href{https://cookbook.ziglang.cc/}{Zig cookbook}\\
	- \href{https://ziggit.dev/}{Zig forum}\\
\end{mycolorbox}
	

%	\begin{mycolorbox}{CtpSapphire}{Template}
%		\textbf{Fetch Dependy:}
%		\begin{minted}[autogobble, breaklines, xleftmargin=1cm,breakafter={/}]{zig}
%		\end{minted}
%	\end{mycolorbox}
	
		%---------------------------------------------
%		\AtNextBibliography{\footnotesize}
		\printbibliography  

	
\end{document}